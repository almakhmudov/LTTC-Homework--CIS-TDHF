\documentclass[12pt]{article}
\usepackage[a4paper, margin=2cm]{geometry}
\usepackage{titlesec}
\usepackage{setspace}
\usepackage{amsmath}

% for tables
\usepackage{array}
\usepackage{booktabs}

% package that forces LaTeX to place an image in the exact position as it determied in the code
\usepackage{float} 

% for a rectangular box
\usepackage[most]{tcolorbox}

% package for multiple columns
\usepackage{multicol}
\setlength{\columnsep}{0.8cm} % separation between columns

\usepackage[utf8]{inputenc}

\usepackage{graphicx}
\usepackage{xcolor}

% For rotating figures, tables, etc. including their captions
\usepackage{rotating}

% small font size in the captions
\usepackage[font=small]{caption}

% header
\usepackage{fancyhdr}
\setlength{\headheight}{15.0pt}
\usepackage{ifthen}
\pagestyle{fancy}
\fancyhf{}
\fancyhead[L]{\ifthenelse{\value{page}=1}{}{Lessons \& Tutorials in Theoretical Chemistry}}
\fancyhead[R]{\ifthenelse{\value{page}=1}{}{Homework}}
\fancyfoot[C]{\thepage}

% bibliography
\usepackage[numbers]{natbib}
\bibliographystyle{unsrtnat}
\usepackage[colorlinks=true, linkcolor=blue, citecolor=blue, urlcolor=blue]{hyperref}
\renewcommand{\bibfont}{\small} % small font in the bibliography

% For DOIs in references
\usepackage{doi}
\renewcommand{\doitext}{DOI: }

% For code snippets
\usepackage{listings}

\definecolor{codegreen}{rgb}{0,0.6,0}
\definecolor{codegray}{rgb}{0.5,0.5,0.5}
\definecolor{codepurple}{rgb}{0.58,0,0.82}
\definecolor{backcolour}{rgb}{0.93,0.93,0.93}

\lstdefinestyle{mystyle}{
    backgroundcolor=\color{backcolour},   
    commentstyle=\color{codegreen},
    keywordstyle=\color{magenta},
    numberstyle=\tiny\color{codegray},
    stringstyle=\color{codepurple},
    basicstyle=\ttfamily\footnotesize,
    breakatwhitespace=false,         
    breaklines=true,                 
    captionpos=b,                    
    keepspaces=true,                 
    numbers=left,                    
    numbersep=5pt,                  
    showspaces=false,                
    showstringspaces=false,
    showtabs=false,                  
    tabsize=2
}
\lstset{style=mystyle}


\title{CIS and TDHF - Homework}
\author{Albert Makhmudov}
\date{19/03/2025}

\begin{document}

\maketitle

\section*{Data Availability}
The source code of this project as well as the example input files are available at the corresponding \href{https://github.com/almakhmudov/LTTC-Homework--CIS-TDHF}{Github page}. There one could also find the example output files and the detailed instructions on how to run and compile the code. The example output files are the outputs of the CIS and TDHF calculations. The source code of this project is written in \texttt{Fortran90}. \\

\section*{Code Overview}

\subsection*{Input}
In order to run the code, one should provide the atom or molecule name and the respective basis set. Currently, Be, He, Ne, and H$_{2}$O are supported as well as the \texttt{cc-pvdz} and \texttt{cc-pvtz} basis sets. The example on how to execute the program can be found in \texttt{README.md} file.

\subsection*{AO to MO transformation}
The AO to MO transformation is performed using the following equation:
\begin{equation}
    (pq|rs) = \sum_{\mu\nu\lambda\sigma} c_{\mu p} c_{\nu q} c_{\lambda r} c_{\sigma s} (\mu\nu|\lambda\sigma)
\end{equation}
where \(c_{\mu p}\) and the others are the coefficients of the MOs in the AO basis and \((\mu\nu|\lambda\sigma)\) are the integrals in the AO basis. The transformation is performed in the \texttt{AO\_to\_MO} subroutine. Below one could find a code snippet. Only part of the code is visualised.

\begin{lstlisting}[language=Fortran, firstnumber=26, label={lst:AOtoMO}, caption=Subroutine for the \texttt{AO\_to\_MO} transformation. Only part of the full code is mentioned.]
! Initialisation
tmp1 = 0.0d0
tmp2 = 0.0d0

! First quarter-transformation
do p = 1, nBas
    do mu = 1, nBas
    tmp1(p,:,:,:) = tmp1(p,:,:,:) + c(mu,p)*ERI_AO(mu,:,:,:)
    end do
end do

! Second quarter-transformation
do p = 1, nBas
    do q = 1, nBas
    do nu = 1, nBas
        tmp2(p,q,:,:) = tmp2(p,q,:,:) + c(nu,q)*tmp1(p,nu,:,:)
    end do
    end do
end do

tmp1 = 0.0d0

! Third quarter-transformation
do p = 1, nBas
    do q = 1, nBas
    do r = 1, nBas
        do la = 1, nBas
        tmp1(p,q,r,:) = tmp1(p,q,r,:) + c(la,r)*tmp2(p,q,la,:)
        end do
    end do
    end do
end do

tmp2 = 0.0d0

! Fourth quarter-transformation
do p = 1, nBas
    do q = 1, nBas
    do r = 1, nBas
        do s = 1, nBas
        do si = 1, nBas
            tmp2(p,q,r,s) = tmp2(p,q,r,s) + c(si,s)*tmp1(p,q,r,si)
        end do
        end do
    end do
    end do
end do

ERI_MO = tmp2
\end{lstlisting}

The transformation is done in an efficient way as discussed in the \citep{goings2013aotomo} to overcome the $N(O^{8})$ scaling. The transformation is done in four steps. This way the scaling is reduced to $N(O^{5})$.

\subsection*{Configuration Interaction with Singles (CIS)}
The CIS matrix is constructed using the following equation:

\begin{equation}
    \mathbf{A} \cdot \mathbf{X}_m = \omega_m \mathbf{X}_m
\end{equation}

where the elements of the matrix \(\mathbf{A}\) are given by

\begin{equation}
    A_{ia, jb} = (\epsilon_a - \epsilon_i) \delta_{ij} \delta_{ab} + 2 (ia|bj) - (ij|ba)
\end{equation}

The excitation energies are obtained by diagonalising the matrix \(\mathbf{A}\). The code snippet for the construction of the CIS matrix is provided below.

\begin{lstlisting}[language=Fortran, firstnumber=24, label={lst:CIS}, caption=Subroutine for the CIS matrix construction. Only part of the full code is mentioned.]
! Construct the CIS matrix
A_matrix = 0.0d0
do ia = 1, nStates
    i = (ia - 1) / nV + 1
    a = mod(ia - 1, nV) + nO + 1
    
    do jb = 1, nStates
        j = (jb - 1) / nV + 1
        b = mod(jb - 1, nV) + nO + 1
        
        ! Diagonal term
        if (i == j .and. a == b) then
            A_matrix(ia,jb) = e(a) - e(i)
        end if
        
        ! Exchange terms
        A_matrix(ia,jb) = A_matrix(ia,jb) + 2.0d0 * ERI_MO(i,b,a,j) &
        - ERI_MO(i,b,j,a)
    end do
end do
\end{lstlisting}

\subsection*{Time-Dependent Hartree-Fock (TDHF)}
To tackle the TDHF problem, the following \(\mathbf{A}\) and \(\mathbf{B}\) matrices are constructed first:
\begin{equation}
    A_{ia, jb} = (\epsilon_a - \epsilon_i) \delta_{ij} \delta_{ab} + 2 (ia|bj) - (ij|ba)
\end{equation}

\begin{equation}
    B_{ia, jb} = 2 (ia|bj) - (ib|ja)
\end{equation}
The \(\mathbf{A}\) matrix is the same as in the CIS case. \\

To obtain the TDHF excitation energies, one should construct and diagonalise the \(\mathbf{C}\) matrix depicted in the following equation:

\begin{equation}
    \mathbf{C} = (\mathbf{A} - \mathbf{B})^{1/2} \cdot (\mathbf{A} + \mathbf{B}) \cdot (\mathbf{A} - \mathbf{B})^{1/2}
\end{equation}

The code snippet for the aforementioned procedures is provided below.

\begin{lstlisting}[language=Fortran, firstnumber=35, label={lst:TDHF}, caption=Subroutine to calculate the TDHF excitation energies. Only part of the full code is mentioned.]
! Construct A and B matrices
do ia = 1, nStates
    i = (ia - 1) / nV + 1
    a = mod(ia - 1, nV) + nO + 1
    
    do jb = 1, nStates
        j = (jb - 1) / nV + 1
        b = mod(jb - 1, nV) + nO + 1
        
        ! A matrix
        ! Diagonal term
        if (i == j .and. a == b) then
            A_matrix(ia,jb) = e(a) - e(i)
        end if
        ! Exchange terms
        A_matrix(ia,jb) = A_matrix(ia,jb) + 2.0d0 * ERI_MO(i,b,a,j) &
        - ERI_MO(i,b,j,a)
        
        ! B matrix
        ! Exchange terms
        B_matrix(ia,jb) = B_matrix(ia,jb) + 2.0d0 * ERI_MO(i,j,a,b) &
        - ERI_MO(i,j,b,a)
    end do
end do

! Compute (A - B) and (A + B)
AminB = A_matrix - B_matrix
AplusB = A_matrix + B_matrix
AminB_temp = AminB

! Compute (A - B)^(-1/2) using matrix diagonalisation
call diagonalize_matrix(nStates, AminB_temp, omega)
do i = 1, nStates
    sqrt_AminB_temp(i,i) = sqrt(omega(i))  ! Square root of eigenvalue
end do
! (A-B)^(-1/2)
AminB = matmul(AminB_temp, matmul(sqrt_AminB_temp, transpose(AminB_temp)))

! Compute C matrix
C_matrix = matmul(AminB, matmul(AplusB, AminB))

! Diagonalize C to get excitation energies
! Reuse omega array
omega = 0.0d0
call diagonalize_matrix(nStates, C_matrix, omega)
omega = sqrt(omega)  ! Excitation energies
\end{lstlisting}

\section*{Results and Discussion}
The reported experimental excitation energies for He, Ne, and Be atoms are 24.587, 21.564, and 9.322 eV, respectively \citep{NIST_ASD}. The calculated excitation energies using the \texttt{cc-pvdz} and \texttt{cc-pvtz} basis sets with the CIS and TDHF methods are presented in Table \ref{tab:excitation_energies}. \\

The results for He and Ne are overestimated while for Be they're underestimated in comparison with the experimental values. The CIS method provides slightly better results compared to the TDHF method in case of the Be atom. On the other hand, the TDHF method shows slightly better results in case of the He atom. The same applies to Ne. This could be due to the lack of correlation effects in the calculations. The correlation effects could be included by using more sophisticated methods such as the flavours of coupled cluster theory. \\

Taking into account the obtained results, one could conclude that both methods could be used to get the solely quantitative picture of the excitation energies. However, the results should be taken with a pinch of salt as the correlation effects are not included in the calculations.

\begin{table}[h!]
    \centering
    \renewcommand{\arraystretch}{1.2}
    \setlength{\tabcolsep}{7pt}
    \begin{tabular}{ccccc}
    \toprule
    & \multicolumn{2}{c}{\textbf{CIS}} & \multicolumn{2}{c}{\textbf{TDHF}} \\ 
    \cmidrule(lr){2-3} \cmidrule(lr){4-5}
    & cc-pvdz    & cc-pvtz    & cc-pvdz     & cc-pvtz    \\ 
    \midrule
    He                   & 51.947     & 31.807     & 51.577      & 31.608     \\
    Ne                   & 49.010     & 33.210     & 48.871      & 33.174     \\
    Be                   & 5.295      & 5.143      & 4.993       & 4.867      \\
    \bottomrule
    \end{tabular}
    \caption{Excitation energies (in eV) for He, Ne, and Be atoms using the \texttt{cc-pvdz} and \texttt{cc-pvtz} basis sets obtained with the implemented CIS and TDHF methods.}
    \label{tab:excitation_energies}
\end{table}

\section*{Generative AI Usage}
In this work, \href{https://chatgpt.com}{ChatGPT} large language model was used to check grammar and spelling of the main body of text, whilst \href{https://www.perplexity.ai}{Perplexity} was utilised for the sake of literature search.

\section*{Acknowledgments}
This project is based on data and instructions provided by Pina Romaniello available at the aforementioned \href{https://github.com/almakhmudov/LTTC-Homework--CIS-TDHF}{Github page}.

\bibliography{references}

\end{document}